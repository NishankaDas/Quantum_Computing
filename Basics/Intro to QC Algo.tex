\documentclass[12pt]{beamer}
\usepackage{braket}
\usepackage{graphicx}
\usepackage{booktabs}
\usepackage{tikz}
\definecolor{myblue}{RGB}{0,102,204}

\usetheme{Madrid}

\setbeamercolor{structure}{fg=myblue}

\title{Quantum Computing}

\subtitle{An Introduction to Quantum Algorithm}
\author{Nishanka Das \and Debanjan Kola}
\date{December 2023}

\begin{document}

\begin{frame}
    \titlepage
\end{frame}

\begin{frame}
    \frametitle{Quantum Algorithm - Query Model}

    \textbf{Query Model:}
    \begin{enumerate}
        \item \textbf{Standard Computation:}
            \begin{itemize}
                \item Input $\rightarrow$ \tikz[baseline=-0.5ex]\node[fill=myblue,anchor=base,rounded corners=2pt,text=white]{Computation}; $\rightarrow$ Output
            \end{itemize}

            The entire input is provided to the computation, most typically as a string of bits with nothing hidden from the computation.
  



    \frametitle{Quantum Algorithm - Query Model}
    \item\textbf{Query model of computation:}
    \begin{center}
        \includegraphics[width=4cm]{Q123.png}
    \end{center}
\end{enumerate}
\end{frame}
\begin{frame}
In the query model of computation , the input is made available in the form of a function , which the computation access by making queries . The input to query problems is represented by a function : $f: \Sigma^n \rightarrow \Sigma^m$ where $n, m \in \mathbb{Z}^+$ and $\Sigma = \{0, 1\}$
\begin{itemize}
\item\textbf{Queries :}
\end{itemize}
To say that a computation makes a query means that it evalues the function f once : $x \in \Sigma^m $ is made available to the computation .\\ 
\par
The efficiency of query algo is measured by counting the number of queries to the input they required .
\end{frame}
\begin{frame}
\begin{itemize}
\item\textbf{Example of query problems :}
\end{itemize}
 \begin{enumerate}
\item\textbf{OR :}

\begin{itemize}
\item\textbf{Input :}$f: \Sigma^n \rightarrow \Sigma$
\item\textbf{Output :} \[
\begin{cases}
    1 & \text{if there exists a string } x \in \Sigma^n \text{ such that } f(x) = 1 \\
    0 & \text{if there is no such string}
\end{cases}
\]
\end{itemize}

\item\textbf{Parity :}
\begin{itemize}
\item\textbf{Input :}$f: \Sigma^n \rightarrow \Sigma$
\item\textbf{Output :} \[
\begin{cases}
    0 & \text{if }f(x) = 1 \text{ for even number of strings } x \in \Sigma^n  \\
    1 & \text{if }f(x) = 1 \text{ for odd number of strings } x \in \Sigma^n
\end{cases}
\]
\end{itemize}
\end{enumerate}
\end{frame}
\begin{frame}[allowframebreaks]
 \begin{enumerate}
\setcounter{enumi}{2}
\item\textbf{Minimum:}
\begin{itemize}
\item\textbf{Input :}$f: \Sigma^n \rightarrow \Sigma^m$
\item\textbf{Output :}
\[
      \begin{split}
        &\text{The string } y \in \{ f(x) = x \in \Sigma^n \} \\
        &\text{that comes first in the lexicographic ordering of } \Sigma^n
      \end{split}
      \]
\end{itemize}
Sometimes we also consider query problems where we have a promise on the input . Inputs that doesn't satisfy the promise are called don't care input .
\item\textbf{Unique search:}
\begin{itemize}
\item\textbf{Input :}$f: \Sigma^n \rightarrow \Sigma$
\item\textbf{Promise :}
\[
\begin{split}
	&\text{There is exactly one string }z \in \Sigma^n \\
 	&\text{ for which f(z) = 1 , with f(x) = 0 for all strings } x \neq z .
 \end{split}
\]
\item\textbf{Output :}\text{ The string z .}
\end{itemize}
\end{enumerate}
\end{frame}
\begin{frame}
\begin{itemize}
\item\textbf{Query Gate :}
 \begin{center}
        \includegraphics[width=6cm]{Q_234.jpg}
    \end{center}
For the circuit model of computation , queries are made by \textbf{query gates} .For the quantum circuit model , we choose a different defination for query gates that makes them unitary - allowing them to be applied to quantum states .
\item\textbf{Definition :}
\[
	\begin{aligned}
         &\text{The query gate } U_f \text{ for any function } f : \Sigma^n \rightarrow \Sigma^m \\
	 &\text{is defined as } U_f(\ket{y} \ket{x}) = \ket{y \oplus f(x)} \ket{x} \quad \forall x\in \Sigma^n \\
	 &\text{and }y \in \Sigma^m. 
	\end{aligned}
\]
\end{itemize}
\end{frame}
\begin{frame}
\begin{itemize}
\item\textbf{Notation :}
 \begin{center}
        \includegraphics[width=6cm]{Q234.jpg}
    \end{center}
The string $y \oplus f(x)$ is the bitwise XOR of y and f(x) . \\
$001 \oplus 101 = 100 $\\
 $\ket{0^m} \rightarrow $ All zero string . 
\item\textbf{Deutsch's Algo :}
Deutsch's problem is parity problem of functions of the form  $f:\Sigma \rightarrow \Sigma$.
\end{itemize}
\end{frame}
\begin{frame}

 
  \begin{table}[ht]
    \centering
    \begin{tabular}{|c|c||c|c||c|c||c|c|}
      \hline
      $a$ & $f(a)$ & $a$ & $f_2(a)$ & $a$ & $f_3(a)$ & $a$ & $f_4(a)$ \\
      \hline
      0 & 0 & 0 & 0 & 0 & 1 & 0 & 1 \\
      1 & 0 & 1 & 1 & 1 & 0 & 1 & 1 \\
      \hline
    \end{tabular}
  \end{table}
$f_1$ and $f_4$ are \textbf{constant} and $f_2$ and $f_3$ are \textbf{balanced} .
\begin{itemize}
\item\textbf{Problem :}
\item\textbf{Input :}
$f:\Sigma \rightarrow \Sigma$
\item\textbf{Output :}
 \[
\begin{cases}
    0 & \text{if f is constant }  \\
    1 & \text{if f is balanced }
\end{cases}
\]
Every classical query algo must make two queries to f to solve this problem learning just one of two bits provids no information about their parity .
\end{itemize}
\end{frame}
\begin{frame}
\begin{itemize}
\item\textbf{Notation :}
 \begin{center}
        \includegraphics[width=9cm]{Q-234.jpg}
    \end{center}
\begin{align*}
\ket{\pi_1} &= \ket{-} \ket{+} \\
&= \left(\frac{1}{\sqrt{2}}\ket{0} - \frac{1}{\sqrt{2}}\ket{1}\right) \frac{1}{\sqrt{2}}\ket{0} + \left(\frac{1}{\sqrt{2}}\ket{0} - \frac{1}{\sqrt{2}}\ket{1}\right) \frac{1}{\sqrt{2}}\ket{1} \\
&= \frac{1}{\sqrt{2}} (\ket{0} - \ket{1}) \ket{0} + \frac{1}{\sqrt{2}} (\ket{0} - \ket{1}) \ket{1}
\end{align*}
\end{itemize}
\end{frame}
\begin{frame}
\begin{itemize}
\item\textbf {$ \ket {\pi_2} $ :}

$ U_f $ gate is performed . According to the definition of the $ U_f $ gate , the value of the function f for the classical state of the top / right most qbit is XORed onto the bottom/left most qbit , which transforms $ \ket { \pi_1} $ to $\ket { \pi_2}$
\begin{align*}
\ket{\pi_2} &= \frac{1}{2}\left(\ket{10\oplus f(0)} - \ket{1\oplus f(0)}\right)\ket{0} \\
&\quad + \frac{1}{2}\left(\ket{10\oplus f(1)} - \ket{1\oplus f(1)}\right)\ket{1} \\
&\quad \left[\ket{10\oplus a}-\ket{1\oplus a} = \left(-1\right)^a \left(\ket{0}-\ket{1}\right)\right] \\
&=\frac{1}{2}\left(-1\right)^{f(0)}\left(\ket{0}-\ket{1}\right)\ket{0}+\frac{1}{2}\left(-1\right)^{f(1)}\left(\ket{0}-\ket{1}\right)\ket{1}\\
&=\ket{-}\left(\frac{\left(-1\right)^{f(0)}\ket{0}+\left(-1\right)^{f(1)}\ket{1}}{\sqrt{2}}\right)
\end{align*}
\end{itemize}
\end{frame}
\begin{frame}
\begin{align*}
&=\left(-1\right)^{f(0)}\ket{-}\left(\frac{\ket{0}+\left(-1\right)^{\left(f(0)\oplus f(1)\right)}\ket{1}}{\sqrt{2}}\right) \\
&=
\begin{cases}
    \left(-1\right)^{f(0)}\hspace{1cm}\ket{-}\ket{+}\hspace{1cm}&f(0)\oplus f(1)=0  \\
    \left(-1\right)^{f(0)}\hspace{1cm}\ket{-}\ket{-}\hspace{1cm}&f(0)\oplus f(1)=1   \\
\end{cases}
\end{align*}
\begin{itemize}
\item \textbf{Phase Kickback:}
\begin{align*}
\ket{b\oplus c}=x^c\ket{b}
U_f\left(\ket{b}\ket{a}\right)=\ket{b\oplus f_a}\ket{a}
&=\left(x^{f(a)}\ket{b}\right)\ket{a} \\
x\ket{-}=-\ket{-} \\
U_f\left(\ket{-}\ket{a}\right)=\mathbf{\left(-1\right)^{f(a)}\ket{-}\ket{a}}
\end{align*}
\item\textbf {Deuctsch-Jozsa Algorithm :}
The Deuctsch Jozsa Algo extends deuctsch's algorithm . To input functions of the form ,
$f:\Sigma^n \rightarrow \Sigma$ for any n$\ge$1
\end{itemize}
\end{frame}

\begin{frame}
\begin{itemize}
\item\textbf {Quantum Circuit :}
 \begin{center}
        \includegraphics[width=9cm]{algo2.jpg}
    \end{center}
\item\textbf{Problem :}
\item\textbf{Input : } $f: \Sigma^n \rightarrow \Sigma$
\item\textbf{Promise :} f is either constant or balanced .
\item\textbf{Output :}  
\[
\begin{cases}
    0 & \text{if f is constant . } \\
    1 & \text{if f is balanced . }
\end{cases}
\]
\end{itemize}
\end{frame}
\begin{frame}
\begin{itemize}
\item \textbf{Analysis:}
    We are taking a Hadamard operation:
    \[
    H\ket{0} = \frac{1}{\sqrt{2}} \ket{0} + \frac{1}{\sqrt{2}} \ket{1}
    \]
    \[
    H\ket{0} = \frac{1}{\sqrt{2}} \ket{0} - \frac{1}{\sqrt{2}} \ket{1}
    \]
    We can express the two equations via one .
 \[
   H\ket{a} = \frac{1}{\sqrt{2}} \ket{0} + \frac{1}{\sqrt{2}}\left(-1\right)^a \ket{1}
 \]
\[
 =\frac{1}{\sqrt{2}}\sum_{b\in\{0,1\}} \left(-1\right)^{ab}\ket{b}
\]
Hadamard operation on each of n qubit :
\[
H^{\oplus n}\ket{X_{n-1}\cdots X_0}
\]
\[
=\left(H\ket{X_{n-1}}\right)\oplus\cdots\oplus\left(H\ket{X_0}\right)
\]
\end{itemize}
\end{frame}
\begin{frame}
\[
= \left(\frac{1}{\sqrt{2}}\sum_{y_{n-1}\in\Sigma} \left(-1\right)^{x_{n-1}y_{n-1}}\ket{y_{n-1}}\right) \oplus \cdots \oplus \left(\frac{1}{\sqrt{2}}\sum_{y_{0}\in\Sigma} \left(-1\right)^{x_{0}y_{0}}\ket{y_{0}}\right)
\]
\[
= \frac{1}{\sqrt{2^n}}\sum_{y_{n-1}\cdots y_{0}\in\Sigma} \left(-1\right)^{x_{n-1}y_{n-1}+\cdots+x_{0}y_{0}}\ket{y_{n-1}\cdots y_{0}}
\]
\begin{itemize}
\item \textbf{Binary Dot Product:}
For binary strings \(x = x_{n-1}\cdots x_{0}\) and \(y = y_{n-1}\cdots y_{0}\),
\[
x\cdot y=x_{n-1}y_{n-1}\oplus\cdots\oplus x_{0}y_{0}
\]
\[
\quad=
\begin{cases}
    1 & \text{if  }x_{n-1}y_{n-1}+\cdots+x_{0}y_{0}\text{  is odd .} \\
    0 & \text{if  }x_{n-1}y_{n-1}+\cdots+x_{0}y_{0}\text{  is even .} 
\end{cases}
\]
\[
H^{\otimes n}\ket{x}=\frac{1}{\sqrt{2}^n}\sum_{y\in\Sigma^n}\left(-1\right)^{x\cdot y}\ket{y}
\]
\end{itemize}
\end{frame}
\begin{frame}
\[
\ket{\pi_1} = \ket{-} \otimes \frac{1}{\sqrt{2}^n}\sum_{x\in\Sigma^n}\ket{x}
\]
\[
\ket{\pi_2} = \ket{-} \otimes \frac{1}{\sqrt{2}^n}\sum_{x\in\Sigma^n}\left(-1\right)^{f(x)}\ket{x}
\]
\[
\ket{\pi_3} = \ket{-} \otimes \frac{1}{\sqrt{2}^n}\sum_{y\in\Sigma^n}\sum_{x\in\Sigma^n}\left(-1\right)^{f(x)+x\cdot y}\ket{y}
\]
The probability of measurements given $y=0^n$ is
\[
p(0^n)=\left|\frac{1}{\sqrt{2}^n}\sum_{x\in\Sigma^n}\left(-1\right)^{f(x)}\right|^2
\]
\[
\begin{cases}
    1 & \text{if f is constant . } \\
    0 & \text{if f is balanced . }
\end{cases}
\]
\end{frame}
\begin{frame}
\begin{itemize}
\item \textbf{Bernstein-Vazirani Problem:}
\item \textbf{Input:}
    $f : \Sigma^n \rightarrow \Sigma$
\item \textbf{Promise:}
    There exists a binary string $s = s_{n-1}\cdots s_0$ for which $f(x) = s \cdot x$ for all $x \in \Sigma^n$
\item \textbf{Output:}
    The string $s$
\[
\ket{\pi_3} = \ket{-} \otimes \frac{1}{\sqrt{2}^n}\sum_{y\in\Sigma^n}\sum_{x\in\Sigma^n} (-1)^{f(x) + x \cdot y}\ket{y} 
\]

\[
= \ket{-} \otimes \frac{1}{\sqrt{2}^n}\sum_{y\in\Sigma^n}\sum_{x\in\Sigma^n} (-1)^{s\cdot x + y\cdot x}\ket{y} 
\]

\[
= \ket{-} \otimes \frac{1}{\sqrt{2}^n}\sum_{y\in\Sigma^n}\sum_{x\in\Sigma^n} (-1)^{\left(s\oplus y\right)\cdot x}\ket{y} 
\]

\[
= \ket{-}\otimes\ket{s}
\]
\end{itemize}
\end{frame}
\begin{frame}
\begin{itemize}
\item \textbf{Simon's Algorithm:}
\item \textbf{Input:}
    $f : \Sigma^n \rightarrow \Sigma^m$
\item \textbf{Promise:}
   $\exists$ a string $s\in\Sigma^n$ such that
\[
\left[f(x)=f(y)\right] \Leftrightarrow \left[(x=y) \lor (x\oplus s)=y\right] \quad \forall x, y \in \Sigma^n
\]
\item \textbf{Output:} The string $s$
\item{case 1 :}
$s=0^n$
the condition in the promise simplifies to 
\[
\left[f(x)=f(y)\right] \Leftrightarrow \left[x=y\right]
\]
this is equivalent to f being one-to-one .
\item{case 2 :}
$s\neq0^n$
the function f must be two-to-one to satisfy the promise
\[
f(x)=f(x\oplus s)
\]
with distinct output strings for each pair .
\end{itemize}
\end{frame}
\begin{frame}
\begin{itemize}
 \item\textbf{Quantum circuit :}
    \begin{center}
        \includegraphics[width=6cm]{algo3..jpg}
    \end{center}
\[
\ket{\pi_1}=\frac{1}{\sqrt{2}^n}\sum_{x\in\Sigma^n}\ket{0^m}\ket{x}
\]
\[
\ket{\pi_2}=\frac{1}{\sqrt{2}^n}\sum_{x\in\Sigma^n}\ket{f(x)}\ket{x}
\]
\[
\ket{\pi_3}=\frac{1}{\sqrt{2}^n}\sum_{x\in\Sigma^n}\ket{f(x)}\oplus\left(\frac{1}{\sqrt{2}^n}\Sigma\left(-1\right)^{x\cdot y}\ket{y}\right)
\]
\end{itemize}
\end{frame}
\begin{frame}
\[
=\frac{1}{2^n}\sum_{y\in\Sigma^n}\sum_{x\in\Sigma^n} (-1)^{x\cdot y}\ket{f(x)}\ket{y}
\]
\[
p(y) = \left\lVert \frac{1}{2^n}\sum_{x\in\Sigma^n} (-1)^{x\cdot y}\ket{f(x)} \right\rVert^2
\]
\[
\begin{cases}
\text{range}(f) = \{ f(x) : x \in \Sigma^n \} \\
f^{-1}(\{ z \}) = \{ x \in \Sigma^n : f(x) = z \}
\end{cases}
\]
\[
= \left\lVert \frac{1}{2^n}\sum_{z\in\text{range}(f)} \left(\sum_{x\in f^{-1}(\{z\})}(-1)^{x\cdot y}\ket{z} \right)\right\rVert^2
\]
\[
=\frac{1}{2^{2n}}\sum_{z\in\text{range}\left(f\right)}\left|\sum_{x\in f^{-1}(\{z\})}(-1)^{x\cdot y}\right|^2
\]
\end{frame}
\begin{frame}
\begin{itemize}
\item{Case 1 :}
$s=0^n$
Because \(f\) is a one-to-one , there a single element \(x\in f^{-1}(\{z\})\) for every \(z\in\) range\((f)\) :
\[
\left|\sum_{x\in f^{-1}(\{z\})}(-1)^{x\cdot y}\right|^2=1
\]
There are \(2^n\) elements in range \(f\) , so
\[
p(y)=\frac{1}{2^{2n}}\cdot 2^n=\frac{1}{2^n}\hspace{1cm}\forall y\in \Sigma^n
\]
\item{Case 2 :}
$s\neq 0^n$
There are two strings in the set \( f^{-1}(\{z\})\) for each \(z\in\) range \(\left(f\right)\) if \( w\in f^{-1}(\{z\})\) either one of them , then \( w\oplus s\) is the other .
\[
\left|\sum_{x\in f^{-1}(\{z\})}(-1)^{x\cdot y}\right|^2=\left|(-1)^{w\cdot y}+(-1)^{\left(w\oplus s\right)\cdot y}\right|^2
\]
\end{itemize}
\end{frame}
\begin{frame}
\[
= \left| 1 + (-1)^{s \cdot y} \right|^2
\]
\[
=
\begin{cases}
    4 \hspace{1cm} s \cdot y = 0 \\
    0 \hspace{1cm} s \cdot y = 1
\end{cases}
\]
There are \(2^{n-1}\) elements in range \((f)\) so:
\[
p(y) = \frac{1}{2^{2n}}\sum_{z\in\text{range}} \left| \sum_{x\in f^{-1}(\{z\})} (-1)^{x \cdot y} \right|^2
\]
\[
=
\begin{cases}
    \frac{1}{2^{n-1}} \hspace{1cm} s \cdot y = 0 \\
    0  \hspace{1.5cm} s \cdot y = 1
\end{cases}
\]
\end{frame}
\begin{frame}
\begin{itemize}
\item\textbf{Integer Factorization:}
\item \textbf{Input:}
An integer \(N \geq 2\)
\item \textbf{Output:}
The prime factorization of \(N\).\\
The Prime Factorization of \(N\) is the list of prime factors of \(N\) and the powers to which they must be raised to obtain \(N\) by multiplication.
\item\textbf{Greatest Common Divisor:}
\item \textbf{Input:} Non-negative integers \(N\) and \(M\) not both 0.
\item \textbf{Output:} The Greatest Common Divisor of \(N\) and \(M\).
\item\textbf{Measuring Computational Cost:}
Abstract overview of computation: \\
\begin{enumerate}
\item{Turing Mechine}
\item{Boolean Circuits}
\item{Quantum Circuits}
\item{Python Programs}
\end{enumerate}
\end{itemize}
\end{frame}
\begin{frame}
\begin{itemize}
\item\textbf{Encoding and input length :}
\item{Inputs and outputs are binary strings .}
\item{Through binary strings we can encode .}
\begin{enumerate}
\item{Numbers}
\item{Vectors}
\item{Matrices}
\item{Graphs}
\item{Description of molecules}
\end{enumerate}
\item{Example :}\\
\begin{tabular}{|c|c|c|}
\hline
Number & Encode & Length \\
\hline
0 & 0 & 1 \\
1 & 1 & 1 \\
2 & 10 & 2 \\
\hline
\end{tabular}
\\Length of binary encoding of N :
\[
\log{N}=
\begin{cases}
1\hspace{2.6cm}N=0\\
1+\log_{2}{N}\hspace{1cm}N\geq1
\end{cases}
\]
\end{itemize}
\end{frame}
\begin{frame}
In general , Input length is the length of the binary String encoding of the input , with respect to whatever encoding scheme has been selected .
\begin{itemize}
\item\textbf{Elementory Operation :}
For circuit based models of computation it is typical that we view each gate as being an elementory operation .
\item{A standard quantum gate set :}
\begin{enumerate}
\item{Single-qubit unary gates from \(X,Y,Z,H,S,S^T,T,T^T\)}
\item{Controlled-NOT gates}
\item{Single-qubit standard basis measurements .}
\end{enumerate}
The Unitary gates in this set are universal-any Unitary operation can be closely approximate by a circuit of the gates .
\item\textbf{A Standard boolean gate set :}
\begin{enumerate}
\item{AND}
\item{OR}
\item{NOT}
\item{FANOUT}
\end{enumerate}
\end{itemize}
\end{frame}
\begin{frame}
\begin{itemize}
\item\textbf{Circuit Size :}
A \textbf{size} of a circuit is the total number of gates it includes, we may write size(c) to refer the size of a circuit .\\
Circuit size corresponds to sequential running time . \\
The \textbf{depth} of a circuit is the maximum number of gates encountered on any path from an input to an output wire .
\item\textbf{Cost as a \(f(x)\) of input length}
Each circuit has a fixed size so we need a family \({c_1,c_2\cdots}\) of circuit to describe an algorithm,typically one circuit for each input length.
\item{Example :}
A classical algorithm for integer factorization could be described by a family of boolean circuits.\\
The cost of such an algorithm is described by a function . \(t(n)=size(C_n)\)
\item\textbf{Asymptotic Notation :}
\item\textbf{Big-O-Notation :}
For two functions \(g(n)\) and \(h(n)\) we write that \(g(n)=O(h(n))\) if there exist a positive real number \(c>0\) and positive integer no such that :
\[
g(n) \leq c\cdot h(n) \quad \forall n \geq n_0
\]
\end{itemize}
\end{frame}
\begin{frame}
\begin{itemize}
\item{Eample :}\\
\begin{enumerate}
\item {Addition of n-bit integer can be computed at cost O(n).}\\
\item{Multiplication of n-bit integer can be computed at cost \(O(n^2)\).}\\
\item{Division of n-bit integer can be computed at cost \(O(n^2)\).}\\
\item{GCD of n-bit integer can be computed at cost \(O(n^2)\).}\\
\item{Moduler exponential cost of n-bit integer can be computed at cost \(O(n^3)\).}\\
\end{enumerate}
\item\textbf{Polynomial vs Exponential cost :}
An algorithm's cost is polynomial if it is \(O(n^b)\) for some fixed constant \(b>0\)\\
An algorithm's cost scales sub exponentially if it is :
\[
O\left(2^{n^{\epsilon}}\right)\quad \forall\epsilon>0
\]
otherwise exponential (or super exponential).
 \item\textbf{Toffoli gates :}
Toffoli gates are controlled-controlled-NOT gate .
    \begin{center}
        \includegraphics[width=7cm]{algo4.jpg}
    \end{center}
\end{itemize}
\end{frame}
\begin{frame}
\begin{itemize}
 \item\textbf{ Simulating Boolean Gates :}
 \item\textbf{ AND gate and OR gate :}
    \begin{center}
        \includegraphics[width=9cm]{algo5.jpg}
    \end{center}
 \item\textbf{ FANOUT gate :}
    \begin{center}
        \includegraphics[width=9cm]{algo6.jpg}
    \end{center}
\end{itemize}
\end{frame}
\begin{frame}
    \begin{center}
        \includegraphics[width=9cm]{algo7.jpg}
    \end{center}

\end{frame}
\begin{frame}{Spectral Theorem}
Suppose U is an NxN unitary matrix \\ 
There exists an orthonormal basis { $\ket{\psi_1}...\ket{\psi_N}$} of vector along with complex numbers \\
$\lambda_1 = e^{2\pi i\theta_1}, ... ,\lambda_N = e^{2\pi i \theta_N}$ such that \\
\begin{center}
$U = \sum_{k=1}^{N}\lambda_k \ket{\psi_k}\bra{\psi_k}$
\end{center}
Each of the vector $\ket{\psi_k}$ is an eigenvector of $U$ having eigenvalue $\lambda_k$
\begin{center}
$U\ket{\psi_k} = \lambda_k\ket{\psi_k} = e^{2\pi i \theta_k} \ket{\psi_k}$
\end{center}
\end{frame}
\begin{frame}{Phase Estimation Problem}
\begin{description}
\item[Input: ] A unitary quantum circuit for an n-qubit operation $U$ and an n qubit quantum state $\ket{\psi}$ 
\item[Promise: ] $\ket{\psi}$ is an eigenvector of $U$
\item[Output: ] An approximation to the number $\theta \in [0,1)$ satisfying $U\ket{\psi}=e^{2\pi i \theta}\ket{\psi}$
\end{description}
We can approximate $\theta$ by fraction 
\begin{center}
$\theta \approx {y/2^m} \quad \forall \quad y\in  \{0,1,...,2^{m-1}\} $
\end{center}
\end{frame}
\begin{frame}{Phase Estimation Procedure}
Approximating phase with low precision using the phase kick back 
\begin{figure}
\includegraphics[width=0.8\linewidth]{img.jpg}
\end{figure}
$\ket{\pi_0} = \ket{\psi}\ket{0}$\\
$\ket{\pi_1} = {1/\sqrt{2}}\ket{\psi}\ket{0} + {1/\sqrt{2}\ket{\psi}\ket{1}}$\\
$\ket{\pi_2}$ = $\ket{\psi}\otimes$ (${1/\sqrt{2}}\ket{0} + {{e^{2\pi i \theta}}/\sqrt{2}}\ket{1}$)\\
$\ket{\pi_3}$ = $\ket{\psi}\otimes ((1+{e^{2\pi i \theta}})/2 + ((1-{e^{2\pi i \theta}})/2)\ket{1}) $
\end{frame}
\begin{frame}{Phase Estimation Procedure}
Measuring Probability \\
\vspace{1cm}
$P_0$ = $\big | \frac{1+e^{2\pi\theta i}}{2} \big |^2$ = $cos^2(\pi\theta)$ \\
\vspace{1cm}
$P_1$ = $\big | \frac{1-e^{2\pi\theta i}}{2} \big |^2$ = $sin^2(\pi\theta)$\\
\end{frame}
\begin{frame}{Doubling the phase}
\includegraphics[scale=0.1]{img2.jpg} \\
\vspace{1cm}
By this circuit the value of $\theta$ is doubled only\\
\end{frame}
\begin{frame}{Two qubit phase estimation}
\includegraphics[scale=0.1]{img3.jpg} \\
$\ket{\pi_1} = \ket{\psi}\otimes \frac{1}{2}{\sum_{a_0=0}^{1}\sum_{a_1=0}^{1}\ket{a_1 a_0}}$\\
$\ket{\pi_2} = \ket{\psi}\otimes \frac{1}{2}{\sum_{a_0=0}^{1}\sum_{a_1=0}^{1}e^{2\pi i a_0 \theta}\ket{a_1 a_0}}$
\end{frame}
\begin{frame}
$\ket{\pi_3} = \ket{\psi}\otimes \frac{1}{2}{\sum_{a_0=0}^{1}\sum_{a_1=0}^{1}e^{2\pi i (a_0+2a_1) \theta}\ket{a_1 a_0}}$\\
\vspace{0.5cm}
$\ket{\pi_3} = \ket{\psi}\otimes \frac{1}{2}{\sum_{x=0}^{3}e^{2\pi i x \theta}\ket{x}}$\\
\vspace{0.5cm}
We will consider a special case where $\theta$ = $\frac{y}{4}$ for y $\in \{0,1,2,3\}$\\
$\therefore \theta \in \{0,\frac{1}{4},\frac{1}{2},\frac{3}{4}\}$\\
\vspace{0.5cm}
$\ket{\phi_y} = \frac{1}{2} \sum_{x=0}^{3}e^{2\pi i x \frac{y}{4}}\ket{x}$\\
\vspace{0.5cm}
\hspace{2cm} where y$\in$ \{0,1,2,3\}
\end{frame}
\begin{frame}
Results :\\
\vspace{1cm}
$\ket{\phi_0} = \frac{1}{2}\ket{0}+\frac{1}{2}\ket{1}+\frac{1}{2}\ket{2}+\frac{1}{2}\ket{3}$\\
\vspace{0.5cm}
$\ket{\phi_1} = \frac{1}{2}\ket{0}+\frac{i}{2}\ket{1}-\frac{1}{2}\ket{2}-\frac{i}{2}\ket{3}$\\
\vspace{0.5cm}
$\ket{\phi_2} = \frac{1}{2}\ket{0}-\frac{1}{2}\ket{1}+\frac{1}{2}\ket{2}-\frac{1}{2}\ket{3}$\\
\vspace{0.5cm}
$\ket{\phi_3} = \frac{1}{2}\ket{0}-\frac{i}{2}\ket{1}-\frac{1}{2}\ket{2}+\frac{i}{2}\ket{3}$\\
\vspace{0.5cm}
\end{frame}
\begin{frame}
Here we are taking the columns of $V$ to be the states $\ket{\phi_0} . . . \ket{\phi_3}$\\
\vspace{1cm}
V = $\frac{1}{2}
\begin{pmatrix}
1 & 1 & 1 & 1\\
1 & i & -1 & -i\\
1 & -1 & 1 & -1\\
1 & -i & -1 & i\\
\end{pmatrix}$\\
\vspace{1cm}
This V $4x4$ matrix is named as $QFT_4$ (Quantum Fourier Transformation)
\end{frame}
\begin{frame}{Quantum Fourier Transformation}
To define the quantum fourier transformation $\omega_N$ , complex number have to be defined first, for positive integer N\\
\vspace{1cm}
$\omega_N = e^{\frac{2\pi i}{N}} = cos(\frac{2\pi}{N}+ i sin(\frac{2\pi}{N}))$\\
\vspace{1cm}
The N dimensional quantum fourier transformation, which is described by NxN matrix whose rows and columns are associated with standard basis state 
$\ket{0}...\ket{N-1}$\\
\vspace{1cm}
$QFT_N = \frac{1}{\sqrt N}(\sum_{x=0}^{N-1}\sum_{x=0}^{N-1}\omega_{N}^{xy}\ket{x}\bra{y})$\\
\end{frame}
\begin{frame}
\begin{figure}[hbtp]
\centering
\includegraphics[scale=0.1]{img4.jpg}
\caption{32 dimentional QFT}
\end{figure}
\end{frame}
\begin{frame}
Computational Cost:\\
\vspace{1cm}
Let $S_m$ denotes the total number of gates in a m qubit QFT\\
For m = 1, a single Hadamard gate is required \\
 For m $\geq$ 2 , these are the gates requires \\
\hspace{2cm} $S_m$ for the $QFT_{m-1}$ gates\\
\hspace{2cm} m-1 controlled  phase gate \\
\hspace{2cm} m-1 swap gate \\
\hspace{2cm} 1 Hadamard gate \\
\[
    \ S_m =
    \begin{cases}
        1, & \text{if } m = 1 \\
        S_{m-1} + 2m-1, & \text{if } m \geq 2
    \end{cases}
\] 

\end{frame}
\begin{frame}
Phase estimation for any  choice of m\\
$P_y$ = $|\frac{1}{2^m}\sum_{x=0}^{2^m-1} e^{2\pi i x (\theta - \frac{y}{2^m})}|^2$ \\
\vspace{1cm}
Best Case:
Suppose $y/2^m$ is the best approximation to $\theta$\\
$|\theta - \frac{y}{2^m}|\leq 2^{-(m+1)}$\\
$P_y \geq \frac{4}{\pi^2} \approx 0.405$ \\
\vspace{1cm}
Worst Case:
Suppose there is a better approximation to $\theta$ between $\frac{y}{2^m}$ to $\theta$\\
$|\theta - \frac{y}{2^m}\geq 2^{-m}|$\\
$P_y \leq \frac{1}{4}$\\
\end{frame}
\begin{frame}{Shor's algorithm}
Order Finding Problem :
\begin{description}
\item[Input: ] Positive integers N and a satisfying $\gcd(N,a)$=1
\item[Output: ] The smallest positive integer r such that $ a^r \equiv 1 $ (mod N)
\end{description}
\end{frame}
\begin{frame}{Factoring by Order-Finding}
If N is odd and not a prime power, order-finding allows us to split N.\\
Iterate the following steps:
\begin{itemize}	
\item Randomly choose a$\in$ {2,...,N-1}
\item Compute d = $gcd(a,N)$ if d $\geq$ 2 then output id d and stop 
\item Compute the order r of a modulo N.
\item If r is even then compute d = $gcd(a^{\frac{r}{2}}-1,N)$ if d $\geq$ 2 then output id d and stop 
\item If this step reached, The method failed
\end{itemize}
There is a 50$\%$ chance of achiving success in this method.
\end{frame}
\begin{frame}{Introduction to Grover's Algo}
Grover's algorithm, which is a quantum algorithm for so-called unstructured search problems that offers a quadratic improvement over classical algorithms. What this means is that Grover's algorithm requires a number of operations on the order of the square-root of the number of operations required to solve unstructured search classically — which is equivalent to saying that classical algorithms for unstructured search must have a cost at least on the order of the square of the cost of Grover's algorithm. 
\end{frame}
\begin{frame}{Unstructured Search}
\begin{itemize}
\item{Input: } $f : \sum^{n} \to \sum$ 
\item{Output: } A string x$\in \sum^{n}$ satisfying f(x) = 1 or "no solution" if no such string exists.
\end{itemize}
This is unstructured search because $f$ is  and there is no promise and also we can't rely on it having a structure that makes finding solution easy. 
\end{frame}
\begin{frame}{Grover's algorithm}
\begin{itemize}
\item Initialize : Set n qubit to state $H^{\otimes n}\ket{0^n}$
\item Iterate : Apply the Grover operation t times 
\item Measure : A standard basis measurement yields a candidate solution.
\end{itemize}
The Grover Operation looks like \hspace{1cm} $G =H^{\otimes n} Z_{OR} H^{\otimes n} Z_f $ 
\includegraphics[scale=0.1]{img5.jpg} 
\end{frame}
\begin{frame}
\begin{itemize}
\item{$Z_f\ket{X}$  : } $Z_f\ket{X}$ = $-1^{f(X)}\ket{X}$
\item{$Z_{OR\ket{X}}$  : } \[
    \ Z_{OR\ket{X}} =
    \begin{cases}
        \ket{X}, & \text{if }  X = 0^n\\
        - \ket{X}, & \text{if } X \neq 0^n
    \end{cases}
\] 

\end{itemize}
\includegraphics[scale=0.1]{img6.jpg} 
\end{frame}
\begin{frame}{Solution and Non-Solution}
We will refer to the n qubit begin used for Grover's Algo as a register Q\\
{Q is initialized to the state $H^{\otimes n}\ket{0^n}$ and Grover operation G is performed iteratively}\\
$A_0 = \{x \in \sum^n : f(x) = 0\}$ \hspace{1cm} Non Solutions\\
$A_1 = \{x \in \sum^n : f(x) = 1\}$ \hspace{1cm} Solutions\\
\vspace{0.5cm}
$\ket{A_0} = \frac{1}{\sqrt{|A_0|}}\sum_{x \in A_0}\ket{x}$\\
$\ket{A_1} = \frac{1}{\sqrt{|A_1|}}\sum_{x \in A_1}\ket{x}$\\
\vspace{0.5cm}
$\ket{u} = H^{\otimes n}\ket{0^n} = \frac{1}{\sqrt{N}}\sum_{x\in \sum^{n}}\ket{x}$\\
$\ket{u} = \sqrt{\frac{|A_0|}{N}}\ket{A_0} + \sqrt{\frac{|A_1|}{N}}\ket{A_1}$
\end{frame}
\begin{frame}{Action of Grover Operation}
$G\ket{A_0} = (2 \ket{u} \bra{u}-1) Z_r \ket{A_0}$\\

$G\ket{A_0}= 2\sqrt{\frac{|A_0|}{N}} \ket{u} - \ket{A_0}$\\

$G\ket{A_0} = \frac{|A_0|-|A_1|}{N} \ket{A_0} + \frac{2 \sqrt{|A_0|.|A_1|}}{N} \ket{A_1} $\\
\vspace{1cm}
$G\ket{A_1} = (2 \ket{u} \bra{u}-1) Z_r \ket{A_1}$\\

$G\ket{A_1}= \ket{A_1} - 2\sqrt{\frac{|A_1|}{N}} \ket{u} $\\
$G\ket{A_1} = - \frac{2 \sqrt{|A_0|.|A_1|}}{N} \ket{A_0} + \frac{|A_0|-|A_1|}{N} \ket{A_1}  $\\
\end{frame}
\begin{frame}{Rotation by an angle}
M = $
\begin{pmatrix}
\frac{|A_0|-|A_1|}{N} &  - \frac{2 \sqrt{|A_0|.|A_1|}}{N}\\
\frac{2 \sqrt{|A_0|.|A_1|}}{N} & \frac{|A_0|-|A_1|}{N}\\
\end{pmatrix}$\\
$ = 
\begin{pmatrix}
\sqrt{\frac{|A_0|}{N}} &  - \sqrt{\frac{|A_1|}{N}}\\
\sqrt{\frac{|A_1|}{N}} & \sqrt{\frac{|A_0|}{N}}\\
\end{pmatrix}^{2}$\\
$ = 
\begin{pmatrix}
cos(2\theta) &  - sin(2\theta)\\
sin(2\theta) & cos(2\theta) \\
\end{pmatrix}^{2}$\\
\vspace{1cm}
$\ket{u} = cos(\theta)\ket{A_0} + sin(\theta)\ket{A_1}$\\
$G \ket{u} = cos(3\theta)\ket{A_0} + sin(3\theta)\ket{A_1}$\\
$G^t \ket{u} = cos((2t+1)\theta)\ket{A_0} + sin((2t+1)\theta)\ket{A_1}$\\
\end{frame}
\begin{frame}{Setting Targate}
Consider a quantum state 
$\alpha\ket{A_0} + \beta\ket{A+1}$\\
\hspace{1cm}
x $\in A_1$ with probability $|\beta|^2$\\
Measuring after t iteration given an outcome x $\in A_1$ with probability \\
\hspace{2cm}
$\sin^2((2t+1)\theta)$\\
\vspace{1cm}
To make this probability close to 1 and minimize t $(2t+1)\theta \approx \frac{\pi}{2} => t =   \lfloor \frac{\pi}{4\theta} \rfloor   $

\end{frame}
\begin{frame}{Unique Search}
For Unique search we have $ s = |A_1| = 1$ and therefore \\
\hspace{2cm}
$\theta = \sin^{-1}(\sqrt{\frac{1}{N}}) \approx \sqrt{\frac{1}{N}} $\\
For N = 128 \\
\hspace{2cm}
 t = 8\\
\includegraphics[scale=0.5]{img7.jpg} 
\end{frame}
\begin{frame}{Multiple Search}
Let's Consider N = 128 and s = 4\\
\hspace{2cm}
$\theta = \sin^{-1}(\sqrt{\frac{s}{N}}) = 0.177...$\\
$ t = \lfloor \frac{\pi}{4\theta} \rfloor = 4$\\
\includegraphics[scale=0.4]{img8.jpg} 
\end{frame}
\begin{frame}{Conclusion}
\begin{itemize}
\item Grover's Algorithm is asymptotically optimal
\vspace{1cm}
\item Grover's Algorithm is broadly applicable 
\vspace{1cm}
\item The technique used in Grover's Algorithm can be generalized 
\end{itemize}
\end{frame}
\end{document}